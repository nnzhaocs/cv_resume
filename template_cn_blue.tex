%# -*- coding:utf-8 -*-
%% start of file `template_en.tex'.
%% Copyright 2006-1008 Xavier Danaux (xdanaux@gmail.com).
%
% This work may be distributed and/or modified under the
% conditions of the LaTeX Project Public License version 1.3c,
% available at http://www.latex-project.org/lppl/.


\documentclass[11pt,a4paper]{moderncv}

\usepackage{fontspec,xunicode}
\setmainfont{Tahoma}
\usepackage[slantfont,boldfont]{xeCJK}
\usepackage{xcolor}                 % replace by the encoding you are using


\setmainfont{Times New Roman}%缺省英文字体.serif是有衬线字体sans serif无衬线字体
\setCJKmainfont[ItalicFont={Kai}, BoldFont={Hei}]{STSong}%衬线字体 缺省中文字体为
\setCJKsansfont{STSong}
\setCJKmonofont{STFangsong}%中文等宽字体
%-----------------------xeCJK下设置中文字体------------------------------%
\setCJKfamilyfont{song}{SimSun}                             %宋体 song
\newcommand{\song}{\CJKfamily{song}}
\setCJKfamilyfont{fs}{FangSong_GB2312}                      %仿宋2312 fs
\newcommand{\fs}{\CJKfamily{fs}}
\setCJKfamilyfont{yh}{Microsoft YaHei}                    %微软雅黑 yh
\newcommand{\yh}{\CJKfamily{yh}}
\setCJKfamilyfont{hei}{SimHei}                              %黑体  hei
\newcommand{\hei}{\CJKfamily{hei}}
\setCJKfamilyfont{hwxh}{STXihei}                                %华文细黑  hwxh
\newcommand{\hwxh}{\CJKfamily{hwxh}}
\setCJKfamilyfont{asong}{Adobe Song Std}                        %Adobe 宋体  asong
\newcommand{\asong}{\CJKfamily{asong}}
\setCJKfamilyfont{ahei}{Adobe Heiti Std}                            %Adobe 黑体  ahei
\newcommand{\ahei}{\CJKfamily{ahei}}
\setCJKfamilyfont{akai}{Adobe Kaiti Std}                            %Adobe 楷体  akai
\newcommand{\akai}{\CJKfamily{akai}}


%------------------------------设置字体大小------------------------%
\newcommand{\chuhao}{\fontsize{42pt}{\baselineskip}\selectfont}     %初号
\newcommand{\xiaochuhao}{\fontsize{36pt}{\baselineskip}\selectfont} %小初号
\newcommand{\yihao}{\fontsize{28pt}{\baselineskip}\selectfont}      %一号
\newcommand{\erhao}{\fontsize{21pt}{\baselineskip}\selectfont}      %二号
\newcommand{\xiaoerhao}{\fontsize{18pt}{\baselineskip}\selectfont}  %小二号
\newcommand{\sanhao}{\fontsize{15.75pt}{\baselineskip}\selectfont}  %三号
\newcommand{\sihao}{\fontsize{14pt}{\baselineskip}\selectfont}         %四号
\newcommand{\xiaosihao}{\fontsize{12pt}{\baselineskip}\selectfont}  %小四号
\newcommand{\wuhao}{\fontsize{10.5pt}{\baselineskip}\selectfont}    %五号
\newcommand{\subwuhao}{\fontsize{10pt}{\baselineskip}\selectfont}    %次五号
\newcommand{\xiaowuhao}{\fontsize{9pt}{\baselineskip}\selectfont}   %小五号
\newcommand{\liuhao}{\fontsize{7.875pt}{\baselineskip}\selectfont}  %六号
\newcommand{\qihao}{\fontsize{5.25pt}{\baselineskip}\selectfont}    %七号


%\usepackage{fontawesome}
% \setCJKmainfont[BoldFont={WenQuanYi Micro Hei/Bold}]{WenQuanYi Micro Hei}
%\defaultfontfeatures{Mapping=tex-text}
%\XeTeXlinebreaklocale "zh"
%\XeTeXlinebreakskip = 0pt plus 1pt minus 0.1pt
% moderncv themes
\moderncvtheme[blue]{classic}                 % optional argument are 'blue' (default), 'orange', 'red', 'green', 'grey' and 'roman' (for roman fonts, instead of sans serif fonts)
%\moderncvtheme[green]{classic}                % idem
%\moderncvtheme[blue,roman]{hht}
% character encoding



% adjust the page margins
\usepackage[scale=0.9]{geometry}
%\setlength{\hintscolumnwidth}{3cm}						% if you want to change the width of the column with the dates
%\AtBeginDocument{\setlength{\maketitlenamewidth}{6cm}}  % only for the classic theme, if you want to change the width of your name placeholder (to leave more space for your address details
\AtBeginDocument{\recomputelengths}                     % required when changes are made to page layout lengths

% personal data
\firstname{赵}
\familyname{楠楠}
\title{}               % optional, remove the line if not wanted
% \address{杭州}{}    % optional, remove the line if not wanted
\address{KWII, 2202 Kraft Dr.}{Blacksburg, VA 24060}{}    % optional, remove the line if not wanted
\phone[mobile]{+1~(540)~998~1060}                      % optional, remove the line if not wanted
%\fax{fax (optional)}                          % optional, remove the line if not wanted
\email{znannan1@vt.edu}                       % optional, remove the line if not wanted
\homepage{http://people.cs.vt.edu/znannan1/}   % optional, remove the line if not wanted
%\social[github]{GitHub: https://github.com/geekplux}
\extrainfo{%
  生日: 1987/10/09 \\
  籍贯: 山东泰安 \\
  性别: 女
}

%\photo[64pt]{avatar.png}                         % '64pt' is the height the picture must be resized to and 'picture' is the name of the picture file; optional, remove the line if not wanted
%\quote{China\TeX 您的LaTeX乐园,TeX\&\LaTeX 王国}                 % optional, remove the line if not wante

%\nopagenumbers{}                             % uncomment to suppress automatic page numbering for CVs longer than one page


%----------------------------------------------------------------------------------
%            content
%----------------------------------------------------------------------------------
\begin{document}
\maketitle
\vspace*{-14mm}

\section{研究方向}
\cvitem{}{容器, 容器编排系统, 分布式存储系统, 文件系统,闪存系统, 磁盘阵列, 和内存系统.}

\section{教育经历}
\cventry{16/08-20/07}{博士}{弗吉尼亚理工大学}{计算机科学}{}{}                % arguments 3 to 6 are optional
\cvlistitem{导师:Ali R. Butt.    \emph{http://research.cs.vt.edu/dssl/}}
\cventry{11/08-16/06}{博士}{华中科技大学}{计算机体系结构}{}{}                % arguments 3 to 6 are optional
\cvlistitem{导师:谢长生}
\cvlistitem{副导师:万继光}
\cventry{09/08–11/06}{硕士}{武汉大学}{}{}{}    
\cventry{05/08–09/06}{本科}{齐鲁工业大学}{}{}{}    

\section{发表论文}
\subsection{已发表的论文}

\cvitem{[CLUSTER'19]}{\textbf{Nannan Zhao}, Vasily Tarasov, Hadeel Albahar, Ali Anwar, Lukas Rupprecht, Dimitrios Skourtis, Amit S. Warke, Mohamed Mohamed, and Ali R. Butt. \textit{Large-Scale Analysis of the Docker Hub Dataset}. IEEE International Conference on Cluster Computing (Cluster'19), Albuquerque, NM, September 2019.}

\cvitem{[CLOUD '19]}{\textbf{Nannan Zhao}, Vasily Tarasov, Ali Anwar, Lukas Rupprecht, Dimitrios Skourtis, Amit Warke, Mohamed Mohamed, and  Ali Butt. \textit{Slimmer: Weight Loss Secrets for Docker Registries}. IEEE 12th International Conference on Cloud Computing (CLOUD), Milan, Italy, 2019.}
% 
\cvitem{[FAST '18]}{Ali Anwar, Mohamed Mohamed, Vasily Tarasov, Michael Littley, Lukas Rupprecht, Yue Cheng, \textbf{Nannan Zhao}, Dimitrios Skourtis, Amit S. Warke, Heiko Ludwig, Dean Hildebrand, and Ali R. Butt. 
	\textit{Improving Docker Registry Design Based on Production Workload Analysis}.
	USENIX Conference on File and Storage Techniques (FAST’18), Okaland, CA.}

\cvitem{[IPDPS '18]}{\textbf{Nannan Zhao}, Ali Anwar, Yue Cheng, Mohammed Salman, Daping Li, Jiguang Wan, Changsheng Xie, Xubin He, Feiyi Wang, and Ali R. Butt.
	\textit{ Chameleon: An Adaptive Wear Balancer for Flash Clusters}.
	IEEE International Parallel and Distributed Processing Symposium (IPDPS’18), Vancouver, Canada.}

\cvitem{[FITEE '16]}{\textbf{Nannan Zhao}, Jiguang Wan, Jun Wang, and Changsheng Xie.
	\textit{A Reliable Power Management for Consistent Hashing based Distributed Key Value Storage Systems}.
	Frontiers of Information Technology \& Electronic Engineering (FITEE), vol. 17, no. 10, pp 994–1007, Oct. 2016.}

\cvitem{[TPDS' 16]}{Dan Luo, Jiguang Wan, Yifeng Zhu, \textbf{Nannan Zhao}, Yifeng Zhu, and Changsheng Xie.
	\textit{Design and Implementation of a Hybrid Shingled Write Disk System}.
	IEEE Transactions on Parallel and Distributed Systems (TPDS’16), vol. 27, no. 4, pp. 1017-1029, Apr. 2016.}

\cvitem{[MSST '15]}{\textbf{Nannan Zhao}, Jiguang Wan, Jun Wang, and Changsheng Xie. 
	\textit{GreenCHT: A Power-Proportional Replication Scheme for Consistent Hashing based Key Value Storage Systems}.
	IEEE International Conference on Massive Storage Systems and Technology (MSST’15, short paper), Santa Clara, CA.}

\cvitem{[TPDS '15]}{Jiguang Wan, Xiaoyang Qu, \textbf{Nannan Zhao}, Jun Wang, and Changsheng Xie. 
	\textit{ThinRAID: Thinning Down RAID Array for Energy Conservation}.
	IEEE Transactions on Parallel and Distributed Systems (TPDS’15). vol. 26, no. 10, pp. 2903-2915, Oct. 2015.}

\cvitem{[Cluster '12]}{Jiguang Wan, \textbf{Nannan Zhao}, Yifeng Zhu, and Changsheng Xie.
	\textit{High Performance and High Capacity Hybrid Shingled- Recording Disk System}.
	IEEE International Conference on Cluster Computing (Cluster’12), Beijing, China.}

\subsection{正在审稿的论文}

\cvitem{[FAST]}{\textbf{Nannan Zhao}, Vasily Tarasov, Hadeel Albahar, Ali Anwar, Lukas Rupprecht, Dimitrios Skourtis, and Ali R. Butt. \textit{Sift: A Docker Registry with Deduplication Support}. USENIX Conference on File and Storage Techniques (FAST’20).}

\cvitem{[TPDS]}{\textbf{Nannan Zhao}, Vasily Tarasov, Hadeel Albahar, Ali Anwar, Lukas Rupprecht, Dimitrios Skourtis, Amit S. Warke, Mohamed Mohamed, and Ali R. Butt. \textit{Large-Scale Analysis of Docker Images}. IEEE Transactions on Parallel and Distributed Systems (TPDS).}

%\section{教育经历}
%
%%\cventry{}{}{}{}{}
%%{博士~~计算机科学  \\导师: Ali R. Butt}  % arguments 3 to 6 can be left empty
%\cventry{11/08-16/06}{华中科技大学}{}{}{}
%{博士~~计算机体系结构 \\导师: 谢长生}
%\cventry{09/08–11/06}{武汉大学}{}{}{}
%{硕士~~印刷工程}
%\cventry{05/08–09/06}{齐鲁工业大学}{}{}{}
%{本科~~印刷工程}
%
%\cventry{17.01-17.12}{弗吉尼亚理工大学}{可爱的项目}{https://keaidexiangmu.com}{}{工作中,
%  我负责了 xxx 的开发和维护,运用 yyy 技术解决了 zzz 的重大问题。积极参与开源社
%  区贡献}
%\cventry{16.01-16.12}{好长的公司名称}{有趣的项
%  目}{http://www.youqudexiangmu.com}{}{工作中,我按照领导的要求编程,做出了让领
%  导满意的作品,为公司做出了贡献。我觉得我还是凑字数吧,编不下去了}
%\cventry{15.01-15.12}{不知道叫什么的公司名称}{不可告人的项
%  目}{http://bukegaoren.com}{}{独立编写了根本编不下去的项目简介,可能还是凑字数
%  比较好,来凑字数吧来凑字数吧来凑字数吧。出色的完成了凑字数的工作,并获得了最佳
%  凑字数员工奖}

\section{实习经历}
%\cventry{05/17–08/17}{IBM 研究院}{美国}{加州, Almaden}{}
\cventry{17/05–17/08}{IBM 研究院--Almaden}{美国加州圣何塞}{云存储组}{}{}                % arguments 3 to 6 are optional
\cvlistitem{项目指导老师: Vasily Tarasov. \emph{https://researcher.watson.ibm.com/researcher/view.php?person=us-vtarasov}}
\cvlistitem{研究项目:大规模Docker镜像分析和IBM数据中心中容器镜像存取负载分析.}

\section{教学经验}

\cventry{16/08–18/12}{教学助理}{弗吉尼亚大学}{}{}{}                % arguments 3 to 6 are optional
\cvlistitem{课程:计算机系统}

\section{专业技能}
%\cvitem{}{Proficient in Linux system/kernel programming and scripting, Hadoop, Spark, and Redis.}
%\cvlistitem{\textbf{Language: } C/C++, Python, Java, Go, Matlab, and bash scripting.}

\cvline{\textbf{系统/平台}}{Linux 系统及内核,Hadoop/Spark, Redis/MySQL/MongoDB, HDFS/Ceph/Sheepdog, 和 Docker/Kubernetes.}
\cvline{\textbf{语言}}{C/C++, Python, Java, JavaScript, Go, 和 Bash.}


\section{科研项目}

\cvline{18/09-现在}{针对容器编排系统的存储优化设计}
\cvlistitem{研究发现在微服务应用中,微服务之间是有关联的。如果在容器部署或者调度时忽视这种关联性,那么系统性能会受到较大影响。}
\cvlistitem{分布式存储系统通常通过容器存储接口(CSI)为容器编排系统提供存储服务。研究发现上层容器编排系统和底层存储系统之间是透明的。
	如果底层存储系统能够获得上层应用的特性(例如关联性)并且利用这些特性来优化数据布局,那么微服务应用内部网络传输开销会大大降低。}
\cvline{18/12-19/09}{Docker镜像存储系统的重删设计}
%\cvlistitem{通过对Docker Hub存储的50TB的容器镜像解压后分析发现,97\%的文件都是冗余,而且如果将冗余文件删除,系统会节省一半的存储空间。}
\cvlistitem{研究发现重删技术会极大影响Docker registry存储系统的性能,特别是容器镜像重构会对镜像传输造成高达200\%额外开销。}
\cvlistitem{提出了:利用用户访问特征来预测并提前进行容器镜像重构来降低重构带来的开销;将重删技术与副本技术相结合来提供不同的重删模式来满足不同用户对性能和存储空间需求。(FAST'20 审稿中)}
\cvline{18/05-18/12}{内存压缩技术设计}
\cvlistitem{研究发现操作系统级的压缩技术当缺页时通过软中断来访问压缩后的数据,这会降低应用程序的性能,而硬件级压缩技术由于无法获得内存利用率的信息会压缩所有的内存页而且其压缩率比较底。}
\cvlistitem{提出一种操作系统和硬件相结合的压缩技术来降低操作系统中断所造成的额外开销同时获得硬件级压缩的压缩性能。(NSF Funding:CCF-1919113)}
\cvline{17/05-18/05}{大规模Docker镜像分析和IBM数据中心中容器镜像存取负载分析} %\textbf{GitHub 1000 stars}}
\cvlistitem{对Docker Hub存储的50TB的容器镜像解压后进行深度分析,特别是对镜像,层,和文件存储特性进行分析,发现97\%的文件都有重复的副本,而且如果将冗余文件删除,系统会节省一半的存储空间。这说明Docker registry镜像存储系统采用的镜像层共享技术的去除冗余数据的能力有限。(Cloud’19和Cluster'19)}
\cvlistitem{对IBM数据中心长达75天的容器镜像存取负载进行分析,主要是针对请求类型分布,访问特性,以及请求响应时间进行分析,并且提出利用用户访问特性来缓存容器镜像来提高性能。(FAST’18)}
\cvline{16/09-17/05}{闪存集群中的损耗均衡设计}
\cvlistitem{研究发现在分布式闪存集群中,I/O 负载的不均衡特性会导致SSD设备之间的损耗不均衡,这会严重影响存储部署的可靠性,性能和寿命,而目前的负载均衡技术没有考虑SSD设备的损耗。}
\cvlistitem{提出了一个损耗均衡技术Chameleon. 它利用副本技术和纠删码技术对SSD设备损耗和性能的不同影响,结合一个基于SSD设备擦写能力的写转移技术来降低分布式闪存集群中的损耗不均衡。(IPDPS’18)}
\cvline{13/06-16/08}{数据中心的能耗问题研究}
\cvlistitem{针对磁盘阵列提出了一个根据负载的变化动态调整磁盘阵列活动磁盘数量的节能方案。(TPDS’16)}
\cvlistitem{基于企业数据中心能耗问题,提出了一个基于CHT的节能技术--GreenCHT。它包括一个多层副本方案,一个可靠的分布式日志存储,和一个能耗模式预测调度器。(MSST’15, FITEE’16)}
\cvline{11/06-14/05}{瓦记录磁盘的容量提升技术研究}
\cvlistitem{针对瓦记录磁盘随机写性能较差的问题,提出了一个新的混合存储架构(磁盘+SSD缓存)和一个波形瓦记录布局来提高系统容量和性能。(Cluster’12, TPDS’16)}
%\cvlistitem{}


%\cvline{\textbf{数据}}{掌握处理数据,掌握数据库}
%\cvline{\textbf{其他}}{熟练使用 Git/Vim/Emacs/Makefile}

%\section{Publications}
%\cvline{已录用}{张三,李四,王麻子 基于 latex 的简历凑字数研究[C]// CVChina. 2017.}

% \subsection{Vocational}
% \cventry{year--year}{Job title}{Employer}{City}{}{Description}                % arguments 3 to 6 are optional
% \cventry{year--year}{Job title}{Employer}{City}{}{Description}                % arguments 3 to 6 are optional
% \subsection{Miscellaneous}
% \cventry{year--year}{Job title}{Employer}{City}{}{Description line 1\newline{}Description line 2}% arguments 3 to 6 are optional

% \section{Languages}
% \cvlanguage{language 1}{Skill level}{Comment}
% \cvlanguage{language 2}{Skill level}{Comment}
% \cvlanguage{language 3}{Skill level}{Comment}

% \section{Computer skills}
% \cvcomputer{category 1}{XXX, YYY, ZZZ}{category 4}{XXX, YYY, ZZZ}
% \cvcomputer{category 2}{XXX, YYY, ZZZ}{category 5}{XXX, YYY, ZZZ}
% \cvcomputer{category 3}{XXX, YYY, ZZZ}{category 6}{XXX, YYY, ZZZ}

% \section{Interests}
% \cvline{篮球}{\small 体力与技巧}
% \cvline{hobby 2}{\small Description}
% \cvline{hobby 3}{\small Description}

% \renewcommand{\listitemsymbol}{-} % change the symbol for lists

% \section{Extra 1}
% \cvlistitem{Item 1}
% \cvlistitem{Item 2}
%\cvlistitem[+]{Item 3}            % optional other symbol% XeLaTeX can use any Mac OS X font. See the setromanfont command below.
% Input to XeLaTeX is full Unicode, so Unicode characters can be typed directly into the source.

% The next lines tell TeXShop to typeset with xelatex, and to open and save the source with Unicode encoding.

%!TEX TS-program = xelatex
%!TEX encoding = UTF-8 Unicode

%\section{Extra 2}
%\cvlistdoubleitem[\Neutral]{Item 1}{Item 4}
%\cvlistdoubleitem[\Neutral]{Item 2}{Item 5}
%\cvlistdoubleitem[\Neutral]{Item 3}{}

%% Publications from a BibTeX file
%\nocite{*}
%\bibliographystyle{plain}
%\bibliography{publications}       % 'publications' is the name of a BibTeX file

% \begin{thebibliography}{}
% \bibitem[]{} 移动增强现实可视化综述[C]. ChinaVis 2017.
% \end{thebibliography}


\end{document}


%% end of file `template_en.tex'.

%%% Local Variables:
%%% mode: latex
%%% TeX-command-extra-options: "-shell-escape"
%%% TeX-master: t
%%% TeX-engine: xetex
%%% End: